% vim: set filetype=tex:ai:et:sw=4:ts=4:sts=4:tw=80
%----------------------------------------------------------------------------------------
%	PACKAGES AND OTHER DOCUMENT CONFIGURATIONS
%----------------------------------------------------------------------------------------

\documentclass{article}

\usepackage[utf8]{inputenc}
\usepackage{fancyhdr} % Required for custom headers
\usepackage{extramarks} % Required for headers and footers
\usepackage{url}
\usepackage{mathtools}
\usepackage{hyperref}

% Margins
\topmargin=-0.45in
\evensidemargin=0in
\oddsidemargin=0in
\textwidth=6.5in
\textheight=9.0in
\headsep=0.25in

\linespread{1.1} % Line spacing

%----------------------------------------------------------------------------------------
%	TITLE
%----------------------------------------------------------------------------------------

\title{
\textmd{IA158 Real Time Systems}\\
\textmd{\textbf{Feedback on presentations}}
}

\author{\textbf{Jan Dupal, Adrian Farmadin, Peter Kotvan, Vít Šesták}}
\date{\today} % Insert date here if you want it to appear below your name


%----------------------------------------------------------------------------------------
\begin{document}

\maketitle

Note: I thought the projects should be more focused on scheduling, what is the
main topic of this subject.

\section*{The Bomb}

I liked this project very much. It incorporated everything specified in the
assignment specifications. I appreciate the ingenuity how the all the sensors
were used to make it harder to defuse the bomb. Regarding the presentation
itself I liked the interaction with the audience. This made the presentation
more interesting and dynamic.

\section*{The Ball Catching Robot}

About this project I found most interesting the solution of sonic sensor
placement. The trick to rotate sensor improved performance significantly and
made task more easy. It's a pity the team experienced a battery problem, which
made the calibration of path memorizing more difficult and the robot wasn't able
to return properly. I think this team done a great job and they fully fulfilled
given project requirements.

\section*{Segway + mobile}

It was great idea to start presentation with the demonstration of the self
balancing segway. This caught the attention of the audience and the effect was
quite remarkable. Considering the presentation itself I would only say that the
presenter could have a little bit better English skills. But this is not a
technical problem. As our team tried to construct a segway too this was
definitely the most interesting project since this was the only segway
successfully balancing even if the controller used for balancing was from the
library.

\section*{Segway accelerometer}

I was surprised, that the team didn't examine any scheduling and controllers. I
think if you build such a complicated system you should know the boundaries what
can the system handle and technologies which you are going to use. In my opinion
the design was a little bit too complicated ( custom sensors ) and this made it
too difficult to realize. 

\end{document}

