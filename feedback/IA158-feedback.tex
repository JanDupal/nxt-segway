% vim: set filetype=tex:ai:et:sw=4:ts=4:sts=4:tw=80
%----------------------------------------------------------------------------------------
%	PACKAGES AND OTHER DOCUMENT CONFIGURATIONS
%----------------------------------------------------------------------------------------

\documentclass{article}

\usepackage[utf8]{inputenc}
\usepackage{fancyhdr} % Required for custom headers
\usepackage{extramarks} % Required for headers and footers
\usepackage{url}
\usepackage{mathtools}
\usepackage{hyperref}

% Margins
\topmargin=-0.45in
\evensidemargin=0in
\oddsidemargin=0in
\textwidth=6.5in
\textheight=9.0in
\headsep=0.25in

\linespread{1.1} % Line spacing

%----------------------------------------------------------------------------------------
%	TITLE
%----------------------------------------------------------------------------------------

\title{
\textmd{IA158 Real Time Systems}\\
\textmd{\textbf{Feedback on presentations}}
}

\author{\textbf{Jan Dupal, Adrian Farmadin, Peter Kotvan, Vít Šesták}}
\date{\today} % Insert date here if you want it to appear below your name


%----------------------------------------------------------------------------------------
\begin{document}

\maketitle

\section*{The Bomb}

I liked this project very much. It incorporated everything specified in the
assignment specifications. I appreciate the ingenuity how the all the sensors
were used to make it harder to defuse the bomb. Regarding the presentation
itself I liked the interaction with the audience. This made the presentation
more interesting and dynamic.

\section*{The Ball Catching Robot}


\section*{Segway (the working one)} 

It was great idea to start presentation with the demonstration of the self
balancing segway. This caught the attention of the audience and the effect was
quite remarkable. Considering the presentation itself I would only say that the
presenter could have a little bit better English shills. But this is not a
technical problem. As our team tried to construct a segway too this was
definitely the most interesting project since this was the only segway
successfully balancing even if the controller used for balancing was from the
library.

\section*{Segway (the other one)}


\end{document}

